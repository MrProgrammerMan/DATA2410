\section{Task 2}

I've opted for something of a functional style here, as that's what I prefer. It turns out python enables higher order functions with lambdas, so I've used those. Some sources say it's better to prefer list comprehension wherever possible, though I'm not sure of this.
The last line is a bit cluttered, but just converts the marks to int.

\begin{minted}[fontsize=\small, linenos, frame=single]{python}
def process_students(lines):
	lines_split = list(map(lambda s: s.split(','), lines))

    # check for correct length and 3rd item number
	l_valid = lambda l: len(l) == 3	and l[2].isdigit()

    # the second check makes the reject output cleaner
	lines_invalid = list(filter(
        lambda l: not l_valid(l) and list_has_non_whitespace(l),
        lines_split
    ))
	lines_valid = filter(l_valid, lines_split)

	if lines_invalid:
		print(f"Rejected the following invalid line(s): {lines_invalid}")

	return list(map(lambda l: (l[0], l[1], int(l[2])), lines_valid))

def list_has_non_whitespace(list):
	return any(len(s.strip()) > 0 for s in list)
\end{minted}